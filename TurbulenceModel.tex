	\subsection{Модель турбулентности}
		Основой правильного численного моделирования течений сжимаемых сред является надёжная и стабильная модель турбулентности. Опыт использования моделей турбулентности показывает, что более сложные модели имеют не слишком большое преимущество над хорошо откалиброванными моделями турбулентной вязкости.
		
		Модель турбулентной вязкости, тем не менее, должна удовлетворять ряду требований для того, чтобы правильно предсказать основные характеристики пограничного слоя. Основным требованием является то, что модель должна ограничивать завышенную генерацию турбулентности в застойных зонах, которая наблюдается в стандартных моделях с двумя уравнениями. В частности, для предсказания теплообмена эти нефизично высокие уровни турбулентности оказывают сильное влияние на скорость передачи тепла в пограничном слое. Для преодоления этого недостатка используются различные модификации стандартных формулировок моделей турбулентности, как то ограничитель генерации, предложенный Ментером (1994) или $k-\varepsilon$ модель Като-Лаундера (1993).
		
		Существенной особенностью пограничного слоя является его отрыв от поверхности при неблагоприятном градиенте давления. Отрыв имеет сильное влияние на характеристики турбулентности, а следовательно, и на теплообмен. SST-модель показала отличные	 возможности предсказания точек отрыва пограничного слоя и наиболее часто используется для анализа течений с теплообменом. Идея SST-модели состоит в сочетании лучших элементов $k-\omega$ и $k-\varepsilon$ моделей при помощи переключающей функции $F_1$, которая равна единице на твёрдой поверхности и нулю вне пограничного слоя. Таким образом, автоматически используется $k-\omega$ модель Уилкокса  в пристеночной области и $k-\varepsilon$ для остальной части потока. Такой подход позволяет использовать эффективную модель Уилкокса в пристенных областях, не имея при этом потенциальных ошибок, связанных с чувствительностю модели Уилкокса в зоне свободного течения. В SST-модели также используется несколько отличное от традиционного выражение для определения турбулентной вязкости, которое может быть интерпретировано, как введение в формулу для $\mu_t$ коэффициента $c_{\mu}$. Эта модификация необходима для того, чтобы правильно предсказать точку отрыва пограничного слоя под действием встречного градиента давления \cite{Menter} .
		Формулировка SST-модели турбулентности, согласно \cite{Menter} (но с учётом поправки на кривизну линий тока), следующая:
		\subparagraph{Уравнение баланса кинетической энергии турбулентности\\}
				\begin{equation}
				\frac{\partial \rho k}{\partial t} + \frac{\partial \rho U_j k}{\partial x_j} = \tilde{P}_k f_{rot} - \beta^* \rho k \omega + \frac{\partial}{\partial x_j}(\Gamma_k \frac{\partial k}{\partial x_j}),
				\end{equation}
				где $f_{rot}$ - поправочный коэффициент Шура-Спалларта к генерации турбулентности, учитывающий криволинейность потока, определяемый в \ref{CC}.
		\subparagraph{Уравнение баланса удельной скорости диссипации\\}
			\begin{equation}
				\frac{\partial \rho \omega}{\partial t} + \frac{\partial U_j \omega}{\partial x_j} = \frac{\gamma}{\nu_t}P_kf_{rot} - \beta\rho\omega^2 + \frac{\partial}{\partial x_j}(\Gamma_{\omega}\frac{\partial \omega}{\partial x_j}) + (1-F_1)2\rho \sigma_{\omega_2}\frac{1}{\omega}\frac{\partial k}{\partial x_j}\frac{\partial \omega}{\partial x_j},
			\end{equation}
			где
			\begin{equation}
				\Gamma_k = \mu + \frac{\mu_t}{\sigma_k}, \quad \Gamma_{\omega} = \mu + \frac{\mu_t}{\sigma_{\omega}}, \quad P_k = \tau_{ij}\frac{\partial U_i}{\partial x_j}
			\end{equation}
			$$
			 \tilde{P}_k = min(P_k, c_1\varepsilon), \quad \mu_t=\rho\frac{a_1 k}{max(a_1\omega, S \cdot F_2)}
			$$
			
			Коэффициент $\phi$ в модели представляет собой функцию от $F_1$: $\phi = F_1\phi_1 + (1-F_1)\phi_2$, где $\phi_1$ и $\phi_2$ соответственно коэффициенты для $k-\omega$ и $k-\varepsilon$ моделей.
			$$
				\sigma_{k1} = 1.176, \quad \sigma_{\omega 1} = 2.0, \quad \gamma_1 = 0.5532 \quad \beta_1=0.075, \quad \beta^*=0.09, \quad c_1 = 10,
			$$
			$$
				\sigma_{k2} = 1.0, \quad \sigma_{\omega 2} = 1.168, \quad \gamma_2 = 0.4403 \quad \beta_2=0.0828, \quad \beta^*=0.09, \quad \kappa = 0.41
			$$
			\begin{eqnarray}
				\nonumber F_1 &=& \tanh{(arg_1^4)}, \quad arg_1 = \min\left[\max\left(\frac{\sqrt{k}}{\beta^*\omega y},\frac{500 \nu}{y^2 \omega} \right), \frac{4\rho\sigma_{\omega 2}k}{CD_{k\omega}y^2}\right] \\ \nonumber CD_{k\omega} &=& \max\left( 2\rho\sigma_{\omega_2}\frac{1}{\omega}\frac{\partial k}{\partial x_j}\frac{\partial \omega}{\partial x_j},10^{-10} \right) \\
				F_2 &=& \tanh{(arg_2^2)}, \quad arg_2 = \max\left(2\frac{\sqrt{k}}{\beta^*\omega y},\frac{500 \nu}{y^2\omega}\right) \\
				\nonumber \tau_{ij} &=& \mu_t\left( \frac{\partial U_i}{\partial x_j} + \frac{\partial U_j}{\partial x_i} - \frac{2}{3}\frac{\partial U_k}{\partial U_k}\right) - \frac{2}{3}\rho k \delta_{ij}
			\end{eqnarray}
			\subparagraph{Турбулентный тепловой поток\\}
			По аналогии с тензором турбулентных напряжений, турбулентный тепловой поток моделируется при помощи турбулентной диффузии:
				\begin{equation}
					\overline{u^{'}_jT^{'}} = - \varepsilon \frac{\partial T}{\partial x_j} = -\frac{\nu_t}{Pr_t}\frac{\partial T}{\partial x_j}, \quad Pr_t = \frac{\nu_t}{\varepsilon_h}
				\end{equation}
				Так, как число Прандтля является свойством вещества, турбулентное число Прандтля полагается постоянным исходя из аналогии между турбулентным теплопереносом и массопереносом. Экспериментальные и теоретические исследования показывают, что турбулентное число Прандтля примерно равно $0.9$.
			\subparagraph{Автоматические пристеночные функции\\}
				В силу того, что пристеночные функции некорректны в случае достаточно подробной сетки, желательно иметь возможность точно разрешать течение в вязком подслое, решая уравнения вплоть до твёрдой поверхности. Идея автоматических пристеночных функций состоит в том, что модель постепенно переходит от формулировки вязкого подслоя к пристеночным функциям в зависимости от подробности расчётной сетки. Уравнение переноса $\omega$ крайне удачный выбор в этом плане, так как оно имеет аналитическое решение как для вязкого подслоя, так и для логарифмического региона. Таким образом, необходимо только определить общую зависимость, основанную на величине $y^{+}$.
				
				Решение для $\omega$ в вязком подслое и логарифмическом регионе:
				\begin{equation}
					\omega_{vis} = \frac{6\nu}{0.075 y^2}, \quad \omega_{log} = \frac{1}{0.3 \kappa}\frac{u_{\tau}}{y}
				\end{equation}
				Это решение может быть переформулировано с учётом $y^{+}$:
				\begin{equation}
						u_{\tau}^{vis} = \frac{U_1}{y^{+}}, \quad u_{\tau}^{log} = \frac{U_1}{\frac{1}{\kappa}\ln(y^{+})+C}, \quad u_{\tau} = {}^4\sqrt{(u_{\tau}^{vis})^4 + (u_{\tau}^{log})^4}
				\end{equation}
				а результирующая функция записана, как
				\begin{equation}
					\omega_1(y^{+})=\sqrt{\omega^2_{vis}(y^{+})+\omega^2_{log}(y^{+})}.
				\end{equation}
				Согласно \cite{Garbarek}, значение $\omega$ на стенке определяется следующим образом:
		\begin{equation}
			\omega_w = 10 \frac{6\nu}{\beta_1 \triangle{y_1}^2}
		\end{equation}
		а $k_w = 0$.
	<<<<<<< HEAD
	\subsubsection{Введение поправки на кривизну линий тока}
		\label{CC}
		Наличие членов, явно учитывающих вклад вращения и кривизны линий тока в уравнениях моделей турбулентности цитируется как фундаментальное преимущество моделей Рейнольдсовых напряжений над более простыми моделями турбулентной вязкости \cite{ShurSpallart}. Внесение эффективных изменений в более простые модели может, тем не менее, иметь широкое применение в силу того \cite{CC2}, что для большого класса вычислительных задач модели Рейнольдсовых напряжений пока не доведены до того состояния, в котором они показали бы свою высокую стабильность и точность расчётов \cite{CC3}.
		
		Влияние вращения и кривизны линий тока на турбулентность проявляется наиболее сильно в двух предельных случаях. В тонких сдвиговых течениях с маленькой, по сравнению со скоростью сдвига, скоростью вращения или слабой кривизной линий тока, наблюдается значительное влияние этих эффектов на уровень турбулентных напряжений \cite{Bradshaw}. Другим крайним случаем является однородное сдвиговое течение во вращающейся области, в котором турбулентные пульсации затухают под влиянием сильного вращения \cite{Speziale}. Другой случай сильного вращения это течение в ядре свободного вихря, моделирование которого также даёт плохие результаты при использовании немодифицированных моделей турбулентности \cite{Govindaraju}.
		
		Для учёта влияния кривизны линий тока в моей работе используется поправка на кривизну линий тока, предложенная для модели Spalart-Almaras в \cite{ShurSpallart} и переформулированная применительно к SST модели в \cite{Smirnov}.
		
		В указанной работе рассматривается сдвиговое течение с базовой скоростью, направленной по оси $x$, а все величины меняются, в основном, в направлении $y$. Пусть $U(y)$ - профиль скорости, и пусть $U_y > 0$ так что $\omega_z < 0$. Состредоточимся на проекции тензора Рейнольдсовых напряжений $-\overline{u^{'}v^{'}}$. Вращение с угловой скоростью $\Omega$ вносит вклад $2\Omega(\overline{{u^{'}}^2}-\overline{{v^{'}}^2})$ в эту величину. Если $\overline{{u^{'}}^2} > \overline{{v^{'}}^2}$, величина касательного напряжения возрастает, и наоборот. Таким образом, генерационный член возникает из-за достаточно тонких особенностей тензора Рейнольдсовых напряжений, которые, конечно, не учитываются в моделях турбулентной вязкости.
		
		В случае криволинейности потока, аналогичный член появляется, если уравнения движения записать в криволинейной системе координат, ориентированной по направлению потока, что приводит к появлению ``эффективной скорости вращения'', равной $U/R$, где $R$ - радиус кривизны линий тока ($R$ принимается положительной в случае вогнутых линий тока). $U/R$ может быть записана, как $\frac{\partial v}{\partial x}$, причём $v$ - скорость, направленная перпендикулярно к линиям тока, а $x$ - направление движения. $\frac{\partial v}{\partial x}$ не является Галилеевым инвариантом так как это частная производная по оси, сонаправленной с вектором скорости.
		
		Неравнозначность $\overline{{u^{'}}^2} > \overline{{v^{'}}^2}$ в тонком сдвиговом течении эквивалетна тому, что главные оси тензора Рейнольдсовых напряжений не сонаправлены с главными осями тензора сдвиговых деформаций. Таким образом, при вращении оси тензора напряжений опережают, или отстают от осей тензора скоростей деформации в зависимости от знака $\Omega$. Авторы статьи выдвигают следующую гипотезу: "под влиянием вращения (или кривизны линий тока) турбулентность усиливается, если главные оси тензора Рейнольдсовых напряжений опережают оси тензора скоростей деформации, и наоборот". Эта гипотеза обобщает эффекты вращения и кривизны, давая $\Omega$ и $U/R$ схожие роли.
		
		В слабо закрученном тонком сдвиговом течении направление движения, направление главных осей тензора Рейнольдсовых напряжений и тензора скоростей деформации меняется с одинаковой скоростью, $U/R$. Оси тензора скоростей деформации инвариантны и могут быть использованы в простой модели турбулентности. Это приводит к фундаментальному соотношению, 
		$$
			\frac{D\alpha}{Dt},
		$$
		где $\alpha$ - угол между главными осями тензора скоростей деформации и осями системы координат. Будучи Лагранжевой производной величины, которая определена относительно инерциальной системы координат, $D\alpha/Dt$ является Галилеевым инвариантом. В однородном вращающемся потоке с деформацией, не зависящей от времени, $D\alpha/Dt = \Omega$. В неоднородном потоке выражение гораздо сложнее, и, следуя авторам, для несжимаемого течения
		\begin{equation}
			\frac{D\alpha}{Dt} = \Omega + \frac{1}{2\left( S_{11}^2 + S_{12}^2 \right)} \left[ S_{11} \frac{DS_{12}}{Dt} - S_{12} \frac{DS_{11}}{Dt} \right],
		\end{equation}
		где $S_{ij}$ -- тензор скоростей деформации во вращающейся системе координат. $S_{21}$ и $S_{22}$ не присутствуют в формуле в силу симметричности тензора.
		
		Промежуточные выкладки для трёхмерного случая слишком громоздки, поэтому, приведём только окончательный вариант поправки на кривизну линий тока, предложенный в \cite{Smirnov}, но без учёта вращения.
=======
	\subparagraph{Поправка на кривизну линий тока\\}
		\label{CC}
		Согласно \cite{Smirnov}, формулировка поправочного члена, учитывающего кривизну линий тока, к модели Ментера следующая:\\
>>>>>>> pif/master
		\begin{equation}
				f_{r1}(r^*,\tilde{r}) = 2r^*\left( \frac{1+C_{r1}}{1+ r^*} \right)\left[ 1-C_{r3}\arctan{(C_{r2}\tilde{r})} \right] - C_{r1},
		\end{equation}
		\begin{equation}
				\tilde{r} = 2\Omega_{ik}S_{kj}\frac{DS_{ij}}{Dt}\frac{1}{\Omega D^3}, \quad D^2 = \max(S^2, 0.09 \omega^2),
		\end{equation}
		$$
				S^2 = 2 S_{ij}S_{ij}, \quad \Omega^2 = 2 \Omega_{ij} \Omega_{ij}, \quad r^* = S/\Omega
		$$
		$$
				C_{r1} = 1, \quad C_{r2} = 2, \quad C_{r3} = 1, \quad f_{rot} = \max[\min(f_{r1},1.25),0]
		$$
		%Поправка на кривизну линий тока

	%Модель турбулентности

	\subsection{OpenFOAM}
		\hspace{2em}OpenFOAM — свободно распространяемый инструментарий вычислительной гидродинамики для операций с полями (скалярными, векторными и тензорными). На сегодняшний день является одним из самых известных приложений с открытым кодом, предназначенных для FVM-вычислений.\cite{openfoam}
		Код OpenFOAM, разработан в Великобритании в компании \textit{OpenCFD, Limited}, и используется многими промышленными предприятиями более 12 лет. Свое название и идеологию построения код берет от предшественника FOAM (Field Operation And Manipulation), который является закрытым и продолжает развиваться параллельно с OpenFOAM. Первоначально, программа предназначалась для прочностных расчетов и в результате многолетнего академического и промышленного развития на сегодняшний момент позволяет решать следующие задачи:
	\begin{itemize}
		\item Прочностные расчеты;
		\item Гидродинамика сжимаемых и несжимаемых сред. Для моделирования турбулентных течений возможно использование RANS и LES - методов. Возможно решение дозвуковых, околозвуковых и сверхзвуковых задач;
		\item Задачи теплопроводности в твёрдом теле;
		\item Течения многофазных сред;
		\item Течения химически реагирующих смесей;
		\item Задачи, связанные с деформацией расчётной сетки;
		\item Распараллеливание расчёта как в кластерных, так и многопроцессорных системах.
	\end{itemize}

	В основе кода лежит набор библиотек, предоставляющих инструменты для решения систем дифференциальных уравнений в частных производных. Рабочим языком кода является C++. В терминах данного языка большинство математических операторов в программном коде уравнений может быть представлено в удобочитаемой форме, а метод дискретизации и решения для каждого оператора может быть выбран уже пользователем в процессе расчёта. Таким образом, в коде полностью инкапсулируются и разделяются понятия расчетной сетки, дискретизации основных уравнений и методов решения алгебраических уравнений.
	%OpenFOAM

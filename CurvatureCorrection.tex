	\subsubsection{Введение поправки на кривизну линий тока}
		\label{CC}
		Наличие членов, явно учитывающих вклад вращения и кривизны линий тока в уравнениях моделей турбулентности цитируется как фундаментальное преимущество моделей Рейнольдсовых напряжений над более простыми моделями турбулентной вязкости \cite{ShurSpallart}. Внесение эффективных изменений в более простые модели может, тем не менее, иметь широкое применение в силу того \cite{CC2}, что для большого класса вычислительных задач модели Рейнольдсовых напряжений пока не доведены до того состояния, в котором они показали бы свою высокую стабильность и точность расчётов \cite{CC3}.
		
		Влияние вращения и кривизны линий тока на турбулентность проявляется наиболее сильно в двух предельных случаях. В тонких сдвиговых течениях с маленькой, по сравнению со скоростью сдвига, скоростью вращения или слабой кривизной линий тока, наблюдается значительное влияние этих эффектов на уровень турбулентных напряжений \cite{Bradshaw}. Другим крайним случаем является однородное сдвиговое течение во вращающейся области, в котором турбулентные пульсации затухают под влиянием сильного вращения \cite{Speziale}. Другой случай сильного вращения это течение в ядре свободного вихря, моделирование которого также показывают плохие результаты при использовании немодифицированных моделей турбулентности \cite{Govindaraju}.
		
		Для учёта влияния кривизны линий тока в данной работе используется поправка на кривизну линий тока, предложенная для модели Spalart-Almaras в \cite{ShurSpallart} и переформулированная применительно к SST модели в \cite{Smirnov}.
		
		В указанной работе рассматривается сдвиговое течение с базовой скоростью, направленной по оси $x$, а все величины меняются, в основном, в направлении $y$. Пусть $U(y)$ - профиль скорости, и пусть $U_y > 0$ так что $\omega_z < 0$. Состредоточимся на проекции тензора Рейнольдсовых напряжений $-\overline{u^{'}v^{'}}$, которая положительна. Вращение с угловой скоростью $\Omega$ вносит вклад $2\Omega(\overline{{u^{'}}^2}-\overline{{v^{'}}^2})$ в эту величину. Если $\overline{{u^{'}}^2} > \overline{{v^{'}}^2}$, величина касательного напряжения возрастает, и наоборот. Таким образом, генерационный член возникает из-за достаточно тонких особенностей тензора Рейнольдсовых напряжений, которые, конечно, не учитываются в моделях турбулентной вязкости.
		
		В случае криволинейности потока, аналогичный член появляется, если уравнения движения записать в криволинейной системе координат, ориентированной по направлению потока, что приводит к появлению "эффективной скорости вращения", равной $U/R$, где $R$ - радиус кривизны линий тока ($R$ принимается положительной в случае вогнутых линий тока). $U/R$ может быть записана, как $\frac{\partial V}{\partial x}$. Нужно заметить, что $\frac{\partial V}{\partial x}$, несмотря на то, что это частная производная от скорости, не является Галилеевым инвариантом так как это частная производная по оси, сонаправленной с вектором скорости.
		
		Неравнозначность $\overline{{u^{'}}^2} > \overline{{v^{'}}^2}$ в тонком сдвиговом течении эквивалетна тому, что главные оси тензора Рейнольдсовых напряжений не сонаправлены с главными осями сдвиговых деформаций (которые повёрнуты на $45^o$ по отношению к осям $(x, y)$), а повёрнуты против часовой стрелки. Таким образом, при вращении оси тензора напряжений опережают, или отстают от осей тензора скоростей деформации в зависимости от знака $\Omega$. Авторы статьи выдвигают следующую гипотезу: "под влиянием вращения (или кривизны линий тока) турбулентность усиливается, если главные оси тензора Рейнольдсовых напряжений опережают оси тензора скоростей деформации, и наоборот". Эта гипотеза обобщает эффекты вращения и кривизны, давая $\Omega$ и $U/R$ одну и ту же роль. 
		
		В слабо закрученном тонком сдвиговом течении направление движения, направление главных осей тензора Рейнольдсовых напряжений и тензора скоростей деформации эволюционируют с одинаковой интенсивностью, $U/R$. Оси тензора скоростей деформации инвариантны и могут быть использованы в простой модели турбулентности. Это приводит к фундаментальному соотношению, 
		$$
			\frac{D\alpha}{Dt},
		$$
		где $\alpha$ - угол между главными осями тензора скоростей деформации и осями системы координат. Будучи Лагранжевой производной величины, которая определена относительно инерциальной системы координат, $D\alpha/Dt$ является Галилеевым инвариантом. В однородном вращающемся потоке с деформацией, не зависящей от времени, $D\alpha/Dt = \Omega$. В неоднородном потоке 
		
		
		\begin{equation}
				f_{r1}(r^*,\tilde{r}) = 2r^*\left( \frac{1+C_{r1}}{1+ r^*} \right)\left[ 1-C_{r3}\arctan{(C_{r2}\tilde{r})} \right] - C_{r1},
		\end{equation}
		\begin{equation}
				\tilde{r} = 2\Omega_{ik}S_{kj}\frac{DS_{ij}}{Dt}\frac{1}{\Omega D^3}, \quad D^2 = \max(S^2, 0.09 \omega^2),
		\end{equation}
		$$
				S^2 = 2 S_{ij}S_{ij}, \quad \Omega^2 = 2 \Omega_{ij} \Omega_{ij}, \quad r^* = S/\Omega
		$$
		$$
				C_{r1} = 1, \quad C_{r2} = 2, \quad C_{r3} = 1, \quad f_{rot} = \max[\min(f_{r1},1.25),0]
		$$
		%Поправка на кривизну линий тока

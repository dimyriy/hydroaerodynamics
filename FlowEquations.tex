\subsection{Уравнения движения}
		\subsubsection{Уравнение баланса массы}
			\begin{equation}
				\frac{\partial \rho}{\partial t} + \frac{\partial}{\partial x_i}(\rho u_i) = 0
			\end{equation}
		\subsubsection{Уравнение баланса импульса}
			\begin{equation}
				\frac{\partial \rho u_i}{\partial t} + \frac{\partial}{\partial x_j}(\rho u_iu_j) = - \frac{\partial p}{\partial x_i} + \frac{\partial {\tau_{ij}}_{eff}}{\partial x_j},
			\end{equation}
			где ${\tau_{ij}}_{eff}$ - тензор вязких напряжений, выражаемый по формуле
			\begin{equation}
				{\tau_{ij}}_{eff} = \mu_{eff}\left( \frac{\partial u_i}{\partial x_j} + \frac{\partial u_j}{\partial x_i} \right) - \frac{2}{3}\mu_{eff}\frac{\partial u_i}{\partial x_j} \delta_{ij} \quad \mu_{eff} = \mu + \mu_{t}
			\end{equation}
		\subsubsection{Уравнение баланса энтальпии}
		\subsubsection{Уравнение состояния}
			\hspace{2em}При расчётах течений сжимаемой жидкости используется модель идеального газа:
			\begin{equation}
				\frac{p}{\rho} = \frac{R}{m}T, \quad m = 28.966 \frac{kg}{mole}
			\end{equation}
			\subsubsection{Зависимость вязкости от температуры}
				\hspace{2em}Зависимость вязкости от температуры выражается формулой Саттерленда для сильно неизотермических течений.
				\begin{equation}
					\mu = \mu_0 \frac{T_0 + C}{T + C_0} \frac{T^{\frac{3}{2}}}{T}, \quad \mu_0 = 1.73 \cdot 10^{-5} kg \cdot m/s, \quad T_0 = 273K, \quad C=110 K
				\end{equation}
				Для течений, температура в которых меняется слабо, вязкость полагается постоянной.
	%Основные уравнения

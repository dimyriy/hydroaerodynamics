\subsection{Уравнения движения}
		\subparagraph{Уравнение баланса массы\\}
			\begin{equation}
				\frac{\partial \rho}{\partial t} + \frac{\partial}{\partial x_i}(\rho u_i) = 0
			\end{equation}
		\subparagraph{Уравнение баланса импульса\\}
			\begin{equation}
				\frac{\partial \rho u_i}{\partial t} + \frac{\partial}{\partial x_j}(\rho u_iu_j) = - \frac{\partial p}{\partial x_i} + \frac{\partial {\tau_{ij}}_{eff}}{\partial x_j} + {S_u}_i,
				\label{flowEqn}
			\end{equation}
			где ${\tau_{ij}}_{eff}$ - тензор вязких напряжений, выражаемый по формуле
			\begin{equation}
				{\tau_{ij}}_{eff} = \mu_{eff}\left( \frac{\partial u_i}{\partial x_j} + \frac{\partial u_j}{\partial x_i} \right) - \frac{2}{3}\mu_{eff}\frac{\partial u_i}{\partial x_j} \delta_{ij} \quad \mu_{eff} = \mu + \mu_{t}
			\end{equation}
		\subparagraph{Уравнение баланса энтальпии\\}
		\begin{equation}
			\frac{\partial \rho h}{\partial t} + \frac{\partial}{\partial x_j} (\rho u_j h) = \frac{\partial p}{\partial t} + \frac{\partial}{\partial x_j} \left((\alpha + \alpha_t) \frac{\partial h}{\partial x_j}\right) - \frac{\partial}{\partial t}\left(\rho \frac{\vec{V}^2}{2}\right) - \frac{\partial}{\partial x_j}\left(\rho u_j \frac{\vec{V}^2}{2}\right) + S_h,
			\label{thermoEqn}
		\end{equation}
		 где $\alpha$ - коэффициент температуропроводности.
		\subparagraph{Уравнение состояния\\}
			\hspace{2em}При расчётах течений сжимаемой жидкости используется модель идеального газа:
			\begin{equation}
				\frac{p}{\rho} = \frac{R}{M}T, \quad M = 28.966 \frac{g}{mole}
			\end{equation}
		
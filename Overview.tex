\section{Обзор существующих исследований}
	\subsection{Экспериментальные исследования}
		\hspace{2em}Существует большое количество работ по экспериментальному исследованию течения с криволинейными линиями тока. Среди них стоит выделить достаточно подробный эксперимент, приведённый в статье Monson et al. \cite{Monson}. Авторы статьи проводят численное и экспериментальное исследование турбулентного течения воздуха в U-образном канале.
	
		Экспериментальному моделированию циклонов также уделено немало внимания. Среди статей, приводящих экспериментальные данные по турбулентному течению в циклонах, нужно отметить детальное исследование течения в циклоне модели Stairmand, описанное в статье J. Dirgo, D. Leith \cite{DirgoLeith}. В этой статье приведены данные для профилей скорости в нескольких сечениях фильтра для большого диапазона рабочих параметров. К сожалению!!!!!!!!!!!!!!!!!!!!!!!!!!!1
	\subsection{Теоретические исследования}
		\hspace{2em}Среди теоретических исследований течения в циклонах особо выделим статью 
	\subsection{Численные исследования}
		\hspace{2em}Численному моделированию течения в циклонах посвящено очень много инженерных исследований.
	%Численные исследования
\newpage
%Обзор существующих моделей

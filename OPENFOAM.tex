	\subparagraph{OpenFOAM\\}
		\hspace{2em}OpenFOAM — свободно распространяемый инструментарий вычислительной гидродинамики для операций с полями (скалярными, векторными и тензорными). На сегодняшний день является одним из самых известных приложений с открытым кодом, предназначенных для FVM-вычислений.\cite{openfoam}
		Код OpenFOAM разработан в Великобритании в компании \textit{OpenCFD}, и используется многими промышленными предприятиями более 12 лет. Свое название и идеологию построения код берет от своего предшественника, FOAM (Field Operation And Manipulation), код которого является закрытым и продолжает развиваться параллельно с OpenFOAM. Первоначально, программа предназначалась для прочностных расчетов и в результате многолетнего академического и промышленного развития на сегодняшний момент позволяет решать задачи динамики сжимаемых и несжимаемых сред с использованием RANS и LES подходов к моделированию турбулентности, включая как дозвуковые, так и сверхзвуковые задачи. Кроме того, OpenFOAM включает в себя большое количество уже готовых солверов для задач динамики жидкости и газа, гетерогенных сред и химически реагирующих смесей, а также задач механики деформируемого твёрдого тела.

	В основе кода лежит набор библиотек, предоставляющих инструменты для решения систем дифференциальных уравнений в частных производных. Рабочим языком кода является C++. В терминах данного языка большинство математических операторов в программном коде уравнений может быть представлено в удобочитаемой форме, а метод дискретизации и решения для каждого оператора может быть выбран уже пользователем в процессе расчёта. Таким образом, в коде полностью инкапсулируются и разделяются понятия расчетной сетки, дискретизации основных уравнений и методов решения алгебраических уравнений.
	%OpenFOAM

<<<<<<< HEAD
\section*{Заключение}
\addcontentsline{toc}{section}{Заключение}

В ходе исследования была рассмотрена задача о турбулентном течении газа с дисперсными включениями в фильтре-циклоне при разных значениях скорости потока и различных диаметрах частиц.

Для проведения расчётов в OpenFOAM был имплементирован поправочный коэффициент Шура-Спалларта для учёта влияния кривизны линий тока на турбулентные характерестики. Верификация модели производилась на задаче о турбулентном течении воздуха в U-образном канале и показала явный положительный эффект на результаты расчётов по сравнению с немодифицированной SST-моделью. Результаты расчётов хорошо согласуются с экспериментальными данными Монсона.

Расчёт циклона также показал положительное влияние введённой поправки на профили скорости. 

Сравнение результатов расчётов, выполненных при помощи введённого в OpenFOAM солвера и Fluent показало отличное соответствие решений друг другу. Анализ влияния дисперсных включений на основной поток позволил заключить, что этим влиянием в исследуемой задаче, по большому счёту, можно пренебречь.

Проведено сравнение результатов расчётов с экспериментальными данными Диргоу и Лейта для степени очистки, которое показало хорошее согласие как с экспериментами, так и с теорией, которую эти эксперименты подтверждают. Показана эффективность циклонов для очищения газа от твёрдых частиц диаметром $\sim 10^{-6}m$. Выявлено, что при уменьшении скорости течения, эффективность рассматриваемой конфигурации циклона заметно понижается. Такая же закономерность имеет место и при уменьшении диаметра частиц. Выяснено, что для фильтрации частиц, диаметром меньше $\approx 10^{-7}m$, циклон непригоден так как степень очистки при этом становится меньше 30\%.
=======
%!TEX root = Dissertation.tex
>>>>>>> pif/master
